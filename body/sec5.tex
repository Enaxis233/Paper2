\section{Quantitative Craig-Simon}
We improve the results of Craig-Simon \cite{craig1983subharmonicity}, on the one hand, to a quantitative sense such that it allow us to estimates $N$, while on the other hand, to be derived only from "large deviation type" (LDT) models. So in fact, we can generalize \cite{craig1983subharmonicity} to a larger variety of models. In this section, we are not limited in the Anderson Model, on the contrary, we consider
\[
 H_\omega u(n)=u(n+1)+u(n-1)+V(\omega)u(n)
\]
where the potential $V$ is an ergodic process in the sense that the index $\omega$ lies in a probability measure space $(\Omega,\mu)$, which supports a group $\tau_x$ of measure preserving ergodic transformations with $V_\omega(x+y)=V_{\tau_y\omega}(x)$, where $\sup\{|V_\omega(x)|,~x\in \mathbb{Z},\omega\in\Omega\}<\infty$.
\begin{definition}
  We say $H_\omega$ satisfies a uniform-LDT condition if for every $\epsilon>0$, there are $\delta>0$, and $M>0$, $c>0$, such that if $\Vert S-T\Vert\leq \delta$, then
  \[
  \mu\left\{\omega\in \Omega:|\frac{1}{n}\log\Vert S_{[0,n],\omega}\Vert-\gamma_n(S)|>\epsilon\right\}\leq Me^{-cn}.
  \]
  where $\gamma_n(S)=\int_{\Omega}\frac{1}{n}\log\Vert S_{[0,n]}\Vert d\mu$.
\end{definition}
By \cite{duarte2016lyapunov}, for those operators that satisfy the uniform-LDT condition, we can conclude the continuity of Lvapunov exponent.

\begin{thm}
  For those operators $H_\omega$ that satisfies the unifotm-LDT condition, and every $\gamma(E)$ is uniformly $L_p$ bounded, the Lyapunov exponent $\gamma(E)$ is continuous.
\end{thm}


\begin{thm}\label{QCS}
  For $H_\omega$ satisfying the uniform-LDT condition, for fixed $\epsilon_0>0$, for a.e. $\omega$ (we denote this set as $\Omega_2=\Omega_2(\epsilon_0)$), there exists $N_2(\omega)$, such that for any $n>N_2(\omega)$, $E\in \sigma$,
  \[
    |P_{[0,n],E,\omega}|\leq e^{(\gamma(E)+\epsilon_0)(n+1)}
  \]
\end{thm}
\begin{remark}
  It's obvious that we can in fact get the result:
  \[
  \max\left\{|P_{[0,n],E,\omega}|,|P_{[-n,0],E,\omega}|,|P_{[n+1,2n+1],E,\omega}|,|P_{[2n+1,3n+1],E,\omega}|\right\}\leq e^{(\gamma(E)+3\epsilon_0)(n+1)}
  \]
  for replacing Theorem 3.3 with a quatitative version.
\end{remark}

% \begin{lemma}
%   If $Q(x)$ is a polynomial of degree $n$, and $x_1,\cdots,x_n$ are $n$ uniformly distributed points in $[x_1,x_n]$. If $Q(x_i)\leq a$ for any $i=1,\cdots,n$, then $Q(x)\leq an^c$ for some $c>0$ and any $x\in[x_1,x_n]$.
% \end{lemma}
Now we prove the Theorem \ref{QCS}.
\begin{proof}
We know that $\sigma$ is compact, so by \eqref{sigma}, $\sigma$ is a finite union of closed intervals. Assume we are dealing with one of them, $[a,a+A]$. By continuity, so uniformly continuity of $\gamma(E)$ on compact set $\sigma$, for $\epsilon_0$, there exists $\delta_0$ such that
\begin{equation}\label{holder}
|\gamma(E_x)-\gamma(E_y)|\leq \epsilon_0,\quad \forall |E_x-E_y|\leq \delta_0.
\end{equation}

Divide the interval $[a,a+A]$ into length $\delta_0$ sub-intervals. There are $K=[A/\delta_0]+1$ of them (the last one may be shorter). Denote them as $I_k$, for $k=1,\cdots, K$. For $I_k$, divide it into $n-1$ equal sub-subintervals with end points $E_{k1,n},\cdots,E_{kn,n}$. Then with any $E_x$, $E_y\in [E_{k1,n},E_{kn,n}]$, $|\gamma(E_x)-\gamma(E_y)|\leq \epsilon_0$.

Since
\[
\mathbb{P}\left(\left\{\omega:  \exists i=1,\cdots,n,~s.t.~|P_{[0,n],E_{ki,n},\omega}|\geq e^{(\gamma(E_{ki,n})+\epsilon_0)(n+1)} \right\}\right)\leq ne^{-\eta_0(n+1)},
\]
by Borel-Cantelli, for $a.e.\omega$, (We denote this set as $\Omega_{k}$,) there exists $N(k, \omega)$ for $I_k$, such that for all $n>N(k,\omega)$,
\[
|P_{[0,n],E_{ki,n},\omega}|\leq e^{(\gamma(E_{ki,n})+\epsilon_0)(n+1)},\quad \forall i=1,\cdots,n.
\]
If we denote $\gamma_{k,n}=\inf_{E\in [E_{k1,n},E_{kn,n}]}{\gamma(E)}$, then by \eqref{holder}
\[
|P_{[0,n],E_{ki,n},\omega}|\leq e^{(\gamma(E_{ki,n})+\epsilon_0)(n+1)}\leq e^{(\gamma_{k,n}+2\epsilon_0)(n+1)}, \quad \forall i=1,\cdots,n.
\]

Let $M$ big enough such that, for any $n>M$, $n^c\leq e^{\epsilon_0(n+1)}$. Thus by Lemma \ref{elementary}, for $E\in[E_{k1,n},E_{kn,n}]$, $n>\max\{N(k,\omega),M\}$,
\[
|P_{[0,n],E,\omega}|\leq n^ce^{(\gamma_{k,n}+2\epsilon_0)(n+1)}\leq n^ce^{(\gamma(E)+2\epsilon_0)(n+1)}\leq e^{(\gamma(E)+3\epsilon_0)(n+1)}
\]
Let $\Omega_2=\bigcap\limits_k \Omega(k)$,  $\tilde{N}(\omega)=\max_k\{N(k,\omega),M\}$, then for any $n>\tilde{N}(\omega)$,
\[
|P_{[0,n],E,\omega}|\leq e^{(\gamma(E)+3\epsilon_0)(n+1)},\quad \forall E\in [0,A]
\]

Use the same methods for $P_{[-n,0],E,\omega}$, $P_{[n+1,2n+1],E,\omega}$, and $P_{[2n+1,3n+1],E,\omega}$. $N_2(\omega)$ being the maximum of $\tilde{N}(\omega)$ for each of them would work for our theorem.
\end{proof}
\begin{remark}\label{N2}
  Similar as remark \ref{N1} and \ref{N3}, we can get $\Omega_2(l)$, $N_2(l,\omega)$ instead. Note $M$ is independent of $l$, and we can then estimate in the same way that, for $a.e.\omega$, (We denote this set as $\Omega_{N_2}$), there exists $L_2=L_2(\omega)$, such that for any $|l|>L_2$, $N_2(l,\omega)\leq \ln^2 |l|$
\end{remark}
\begin{remark}
  For LD implies continuity of $\gamma$.
\end{remark}
