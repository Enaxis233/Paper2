\section{Preliminaries}
\begin{definition}
    We call $E$ a generalized eigenvalue ($g.e.$), if there exists a nonzero polynomially bounded function $\Psi(n)$ such that $H_\omega\Psi=E\Psi$. We call $\Psi(n)$ a generalized eigenfunction.
\end{definition}
Since the set of g.e. supports the spectral measure of $H_\omega$, we only need to show:
\begin{thm}\label{thm1}
  For a.e. $\omega$, for every g.e. E, the corresponding generalized eigenfunction $\Psi_{\omega,E}(n)$ decays exponentially in $n$.
\end{thm}

For $[a,b]$ an interval, $a,b\in\mathbb{Z}$, define $H_{[a,b],\omega}$ to be operator $H_\omega$ resticted to $[a,b]$ with zero boundary condition outside $[a,b]$. Note that it can be expressed as a "$b-a+1$"-dimensional matrix.
The Green's function defined on $[a,b]$ for $H_\omega$ with energy $E\notin\sigma_{[a,b],\omega}$ is
  \[
    G_{[a,b],E,\omega}=(H_{[a,b],\omega}-E)^{-1}
  \]
Note that this can also be expressed as a "$b-a+1$"-dimensional matrix. Denote its $(x,y)$ entry as $G_{[a,b],E,\omega}(x,y)$.

It is well known that
  \begin{equation}\label{possion}
    \Psi(x)=-G_{[a,b],E,\omega}(x,a)\Psi(a-1)-G_{[a,b],E,\omega}(x,b)\Psi(b+1),\quad x\in[a,b]
  \end{equation}
and we have
\begin{equation}\label{sigma}
\sigma:=\sigma(H_{\omega})=[-2,2]+S\quad a.e.\omega.
\end{equation}

% If one can get that, the Green's function near $n$, say, for example on $[n-k,n+k]$, is decaying somehow exponentially in $n$ as $n$ growing, then since $\Psi$ on the  right-hand-side is polynomially bounded, $\Psi(n)$ on the left-hand-side will decay exponentially in $n$, too.

% inspires us to define "regular and singular".

% \begin{definition}
% $\mathcal{A}_{[a,b]}=\{[a,b],[a,b-1],[a+1,b],[a+1,b-1]\}$
% \end{definition}
\begin{definition}
   For $c>0, n\in\mathbb{Z}$, we say $x\in\mathbb{Z}~$ is $(c,n,E,\omega)$-regular, if
  \[
    G_{[x-n,x+n],E,\omega}(x,x-n) \leq e^{-cn}
  \]
  % i.e. if $A=[x_1,x_2]$
  \[
    G_{[x-n,x+n],E,\omega}(x,x+n) \leq e^{-cn}
  \]
  Otherwise, we call it $(c,n,E,\omega)$-singular.
\end{definition}

By \eqref{possion} and definition 2, Theorem  \ref{thm1} follows from
\begin{thm}\label{thm2}
  There exists $ \Omega_0$ with $\mathbb{P}(\Omega_0)=1$, such that for every $ \tilde{\omega}\in\Omega_0$, for any g.e.$\tilde{E}$ of $H_{\tilde{\omega}}$, there exist $ N=N(\tilde{E},\tilde{\omega}),C=C(\tilde{E})$, for every $ n>N$, $2n,~2n+1$ are $(C,n,\tilde{E},\tilde{\omega})$ regular.
\end{thm}
% \begin{remark}
%   It's similar for even terms. We omit them only because of notation reasons.
% \end{remark}
% \begin{remark}
%   Because if we achieve this, denote the corresponding polynomially bounded generalized eigenfunction as $\Psi(n)=\Psi_{\tilde{\omega},\tilde{E}}(n)\leq M(1+n)^p$, for $p>0$. Then for every $ n>N$, let $A_n=[n+1,3n+1]$
%   \[
%   \left\vert G_{A_n,E,\omega}(x,\partial A_n)\right\vert \leq e^{-C\abs{x-\partial A_n}}
%   \]
%   so by eqution \ref{possion}, and $\abs{x-\partial A_n}\geq n-1$.
%   \[
%   \abs{\Psi(2n+1)}\leq Me^{-Cn}(1+3n+1)^p
%   \]
%   So for large enough $n$, $\Psi(2n+1)$ decays exponentially in $n$. Similarly for even terms and we will get Theorem \ref{thm1}.
% \end{remark}

Some other basic settings are below. Denote
\[
  P_{[a,b],E,\omega}=det(H_{[a,b],E,\omega}-E)
\]
If $a=b$, let $  P_{[a,b],E,\omega}=1$. Then
\begin{equation}\label{A}
  \left\vert G_{[a,b],E,\omega}(x,y)\right\vert=\frac{\left\vert P_{[a,x-1],E,\omega}P_{[y+1,b],E,\omega}\right\vert}{\left\vert P_{[a,b],E,\omega}\right\vert},\quad x\leq y
\end{equation}
%   \[
%   \left\vert G_{[a,b](x,y),E,\omega}\right\vert=\frac{\left\vert P_{[a,y-1],E,\omega}P_{[x+1,b],E,\omega}\right\vert}{\left\vert P_{[a,b],E,\omega}\right\vert},\quad x\geq y?
% \]
If we denote the transfer matrix $T_{[a,b],E,\omega}$ as the matrix such that
\[
\left(
\begin{array}{c}
  \Psi(b)\\
  \Psi(b-1)\\
\end{array}
\right)
=T_{[a,b],E,\omega} \left(
\begin{array}{c}
  \Psi(a)\\
  \Psi(a-1)\\
\end{array}
\right)
\]
then
\[
T_{[a,b],E,\omega}=\left(
  \begin{array}{cc}
    P_{[a,b],E,\omega} & -P_{[a+1,b],E,\omega}\\
    P_{[a,b-1],E,\omega} & -P_{[a+1,b-1],E,\omega}\\
  \end{array}
  \right)
\]
The Lyapunov exponent is given by
  \[
    \gamma(E)=\lim_{n\to\infty}\frac{1}{n}\int_0^1 \log\Vert T_{[0,n],E,\omega}\Vert d\mathbb{P}(\omega)=\lim_{n\rightarrow\infty}\frac{1}{n} \log\Vert T_{[0,n],E,\omega}\Vert, \quad a.e.\omega.
  \]
Let $\nu=\inf\limits_{E\in \sigma}\gamma(E)>0$.

We introduce the large deviation theorem here without proof.
\cite{tsay1999some}
\begin{lemma}[Large deviation estimates]\label{ldt lemma}
  For any $\epsilon>0$, there exists $\eta=\eta(\epsilon)>0$ such that, there exists $N_0=N_0(\epsilon)$, for every $ b-a>N_0$
  \[
  \mu \left\{ \omega:\left\vert \frac{1}{b-a+1} \log\Vert P_{[a,b],E,\omega}\Vert-\gamma(E) \right\vert\geq\epsilon
   \right\} \leq e^{-\eta (b-a+1)}
  \]
\end{lemma}
%
% Denote
% \begin{equation}\label{B+}
%     B_{[a,b],\epsilon}^{+} =\left\{(E,\omega): |P_{[a,b],E,\omega}|\geq e^{(\gamma(E)+\epsilon)(b-a+1)}\right\}
% \end{equation}
% \begin{equation}\label{B-}
%     B_{[a,b],\epsilon}^{-} =\left\{(E,\omega): |P_{[a,b],E,\omega}|\leq e^{(\gamma(E)-\epsilon)(b-a+1)}\right\}\
% \end{equation}
% and denote $B_{[a,b],\epsilon,E}^{\pm}=\{\omega:(E,\omega)\in B_{[a,b],\epsilon}^{\pm}\}$, $B_{[a,b],\epsilon,\omega}^{\pm}=\{E:(E,\omega)\in B_{[a,b],\epsilon}^{\pm}\}$,
% $B_{[a,b],*}=B_{[a,b],*}^+\cup B_{[a,b],*}-$.
%
% Let $E_{j,(\omega_a,\cdots,\omega_b)}$ be the eigenvalue of $H_{[a,b],\omega}$ with $\omega|_{[a,b]}=(\omega_a,\cdots,\omega_b)$.

% The main technique is to find the proper set $\Omega_0$. Roughly speaking, the idea is: a point $x$ is $(C,k,E,\omega)$-singular means the corresponding pair $(E,\omega)$ is in some "bad" "large deviation set" for operators restricted near $x$, say, $H_{[x-k,x+k]}$. We can then pick proper set $\Omega_1$ such that these bad sets have small measures. Then pick $\Omega_2$ such that $E$ is not only in these bad sets, but also stay very close to eigenvalues of $H_{[x-k,x+k]}$ which are also in these bad sets. In this case, if we pick $\tilde{\omega}\in\Omega_1\cap\Omega_2$ and if $0$ and $2n+1$ are both $(C,n,\tilde{E},\tilde{\omega})$-singular, then $\tilde{E}$ will  be close to both eigenvalues $E_{i,n}$ of $H_{[-n,n]}$ near $0$ and eigenvalues $E_{j,n}$ of $H_{[n+1,3n+1]}$ near $2n+1$. So now $\tilde{E}$ is in bad sets for both intervals and are closed to both e.v., where each e.v. only belongs to bad sets for their own intervals. Which leads to a contradiction because the bad sets are so small that we can pick $\Omega_3$ by Borel Cantelli, such that eventually all the eigenvalues of $[n+1,3n+1]$ can't stay in bad set for $[-n,n]$,


% In order to find the set $\Omega_0$, we give these three Theorems, each provides us some nice properties we need to make $2n+1$ $(C,n,\tilde{E},\tilde{\omega})$-regular.
%
% \begin{thm}
%   for $a.e.\omega$ (denote the set as $\Omega_1$), there exists $ N_1=N_1(\omega)$, such that for every $ n>N_1$,
%   \[
%   m(B_{[-n,n],\omega}^{-})\leq Ce^{-(\eta-\delta)(2n+1)}
%   \]
% \end{thm}

Finally, we provide an estimates on polynomials interpolation.
%
% We begin with an elementary Lemma:
\begin{lemma}\label{elementary}
  If $Q(x)$ is a polynomial of degree $n-1$ on and $x_1,\cdots,x_n$ are distributed as
  $x_i=\cos{\frac{2\pi(i+\theta)}{n}}$, if $Q(x_i)\leq a^n, \forall i$, then $Q(x)\leq Cna^n$, where $C$ is a constant, $x\in[x_1,x_n]$.
\end{lemma}
\begin{proof}
  By Lagrange Interpolation, we have
  \begin{equation}
    Q(x)=\sum_{i=1}^{n}Q(x_i)\prod_{j\neq i}\frac{x-x_j}{x_i-x_j}
  \end{equation}
  Note that
  \[
  \sum_{j\neq i}\ln|x_i-x_j|=\sum_{j\neq i}\left\{\ln\left|\sin\frac{\pi(i+j+2\theta)}{n}\right|+\ln\left|\sin\frac{\pi(i-j)}{n}\right|+\ln 2\right\}= A+B+C
  \]
  For $B$, one use lemma 9.6 from \cite{avila2009ten} and get
  \[
    B\geq \ln n+\ln (2/\pi)-(n-1)\ln 2.
  \]
  For $A$, if $j=j_0$ reach the minimum term of $\ln|\sin\frac{\pi(i+j+2\theta)}{n}|$, then
  \[
    A\geq \ln n+\ln (2/\pi) -(n-1)\ln 2-\ln\left|\sin\frac{\pi(2i+2\theta)}{n}\right|+\ln\left|\sin\frac{\pi(i+j_0+2\theta)}{n}\right|
  \]
  Choose $0<\theta<1/2$, then
  \[
  \frac{|\sin\frac{\pi(2i+2\theta)}{n}|}{|\sin\frac{\pi(i+j_0+2\theta)}{n}|}
  = \frac{|\sin\frac{\pi(2i+2\theta)}{n}|}{|\sin\frac{\pi\cdot 2\theta}{n}|}\leq \frac{1}{|\sin\frac{\pi\cdot 2\theta}{n}|}= O(n)
  \]
  So
  \[
    \sum_{j\neq i}\ln|x_i-x_j|\geq -(n-1)\ln 2+\ln n+C
  \]
  Write $x=\cos\frac{2\pi a}{n}$ and use the other half result of lemma 9.6 from \cite{avila2009ten}, one get
  \[
    \sum_{j\neq i}\ln |x-x_j|\leq -(n-1)\ln 2+ 2\ln n+ C'
  \]
  So
  \[
  \prod_{j\neq i}\frac{x-x_j}{x_i-x_j}\leq Cn
  \]
  \[
  Q(x)\leq Cna^n
  \]
\end{proof}
