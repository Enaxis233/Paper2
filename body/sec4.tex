\section{Proof of Theorem 2.2}
We will only provide a proof that $2n+1$ is $(c, n, E,\omega)$-regular, the argument for $2n$ being similar.
\begin{proof}
Let $\epsilon$ be small enough such that
  \begin{equation}\label{epsilon1}
    \epsilon<\min\{(\eta_0-\delta_0)/3,\nu\}.
  \end{equation}
Now let
  \[L:=e^{(\eta_0-\delta_0-\epsilon)}>1,\]
and note that since $V$ is bounded, by \eqref{sigma} we have there exists $ M>0$, such that
\[
|P_{[a,b],E,\omega}|<M^{(b-a+1)},\quad \forall E\in\sigma,\omega
\]
Pick $K$ big enough such that
  \[M^{\frac{1}{K}}<L\]
Let $\sigma>0$ be such that
\begin{equation}\label{K}
M^{\frac{1}{K}}\leq L-\sigma<L
\end{equation}
Let $\Omega_0=\Omega_1\cap\Omega_2\cap\Omega_3(\epsilon,K)$. Pick $\tilde{\omega}\in\Omega_0$, and take $\tilde{E}$ a $g.e.$ for $H_{\tilde{\omega}}$.
Without loss of generality assume $\Psi(0)\neq 0$. Then there exists $ N_4$, such that for every $ n>N_4$, 0 is $(\gamma(\tilde{E})-8\epsilon_0,n,\tilde{E},\tilde{\omega})$-singular.

% Assume $2n+1$ is not eventually $(\gamma(\tilde{E})-8\epsilon_0,n,\tilde{E},\tilde{\omega})$-regular, then there exists $ \{n_k\}$ with $n_k\to\infty$, such that $2n_k+1$ is  $(\gamma(\tilde{E})-3\epsilon_0,n_k,\tilde{E},\tilde{\omega})$-singular.

For $n>N_0=\max\{N_1(\tilde{\omega}),N_2(\tilde{\omega},\tilde{E}),N_3(\tilde{\omega}),N_4(\tilde{\omega},\tilde{E})\}$, assume $2n+1$ is $(\gamma(\tilde{E})-8\epsilon_0,n,\tilde{E},\tilde{\omega})$-singular.
% WLOG use $\{n\}$ istead of $\{n_k\}$, we have that
% \begin{itemize}
 Then both $0$ and $2n+1$ is $(\gamma(\tilde{E})-8\epsilon_0,n,\tilde{E},\tilde{\omega})$-singular. So by Lemma \ref{lemma1},
  $\tilde{E}\in B_{[n+1,3n+1],\epsilon_0,\tilde{\omega}}^-\cup B_{[n+1,2n+1],\epsilon_0,\tilde{\omega}}^+ \cup B_{[2n+1,3n+1],\epsilon_0,\tilde{\omega}}^+ $.
  By Corollary \ref{omega2} and \eqref{B+}, $\tilde{E}\notin B_{[n+1,2n+1],\epsilon_0,\tilde{\omega}}^+\cup B_{[2n+1,3n+1],\epsilon_0,\tilde{\omega}}^+$, so it can only lie in $B_{[n+1,3n+1],\epsilon_0,\tilde{\omega}}^-$.

  % But  by \eqref{B-}.
  % \begin{equation}\label{bad}
  %   B_{[n+1,3n+1],\epsilon_0,\tilde{\omega}}^-= \left\{E:|P_{[n+1,3n+1],\epsilon,E,\tilde{\omega}}|\leq e^{(\gamma(E)-\epsilon_0)(2n+1)}\right\}
  % \end{equation}
   Note that in \eqref{B-}, $P_{[n+1,3n+1],\epsilon,\epsilon_0,E,\tilde{\omega}}$ is a polynomial in $E$ that has $2n+1$ real zeros (eigenvalues of $H_{[n+1,3n+1],\tilde{\omega}}$), which are all in $B=B_{[n+1,3n+1],\epsilon,\tilde{\omega}}$. Thus $B$ contains less than $2n+1$ intervals near the eigenvalues. $\tilde{E}$ should lie in one of them. By Theorem \ref{omega1}, $m(B)\leq Ce^{-(\eta_0-\delta_0)(2n+1)}$. So there is some e.v. $E_{j,[n+1,3n+1],\tilde{\omega}}$ of $H_{[n+1,3n+1],\omega}$ such that
   \[
   \vert\tilde{E}-E_{j,[n+1,3n+1],\tilde{\omega}}\vert\leq e^{-(\eta_0-\delta_0)(2n+1)}
   \]
  By the same argument, there exists $ E_{i,[-n,n],\tilde{\omega}}$, such that
   \[
   \vert\tilde{E}-E_{i,[-n,n],\tilde{\omega}}\vert\leq e^{-(\eta_0-\delta_0)(2n+1)}
   \]
   Thus $\vert E_{i,[-n,n],\tilde{\omega}}-E_{j,[n+1,3n+1],\tilde{\omega}}\vert\leq 2e^{-(\eta_0-\delta_0)(2n+1)}$. However, by Theorem \ref{omega3}, one has $E_{j,[n+1,3n+1],\tilde{\omega}}\notin B_{[-n,n],\epsilon,\tilde{\omega}}$, while $E_{i,[-n,n],\tilde{\omega}}\in B_{[-n,n],\epsilon,\tilde{\omega}}$
   This will give us a contradiction below.\\
% \end{itemize}
% ~\\
Since $\vert E_{i,[-n,n],\tilde{\omega}}-E_{j,[n+1,3n+1],\tilde{\omega}}\vert\leq 2e^{-(\eta_0-\delta_0)(2n+1)}$ and $E_{i,[-n,n],\tilde{\omega}}$ is the e.v. of $H_{[-n,n],\tilde{\omega}}$,
\[
  \left\Vert G_{[-n,n],E_{j,[n+1,3n+1],\tilde{\omega}},\tilde{\omega}}\right\Vert\geq \frac{1}{2}e^{(\eta_0-\delta_0)(2n+1)}
\]
Thus there exist $ y_{1},y_{2}\in [-n,n]$ and such that
% Need fix
\[
  \left\vert G_{[-n,n],E_{j,[n+1,3n+1],\tilde{\omega}},\tilde{\omega}}(y_{1},y_{2})\right\vert\geq \frac{1}{2n}e^{(\eta_0-\delta_0)(2n+1)}
\]
Let $E_j=E_{j,[n+1,3n+1],\tilde{\omega}}$. We have $E_j\notin B_{[-n,n],\epsilon,\tilde{\omega}}$, thus
\[
\vert P_{[-n,n],\epsilon,E_j,\tilde{\omega}}\vert\geq e^{(\gamma(E_j)-\epsilon)(2n+1)}
\]
so by \eqref{A},
\begin{equation}\label{last}
  \left\Vert P_{[-n,y_{1}],\epsilon,E_j,\tilde{\omega}}P_{[y_{2},n],\epsilon,E_j,\tilde{\omega}}\right\Vert\geq\frac{1}{2n}e^{(\eta_0-\delta_0)(2n+1)}e^{(\gamma(E_j)-\epsilon)(2n+1)}
\end{equation}
% Let $M= sup\{|V|+|E_j|+2\}$, where $|V|$ is assumed bounded, $E_i,E_j$ are bounded because they are close to $E\in I$.\\
% Then pick $\epsilon$ small enough in Theorem \ref{omega3} such that
%   \begin{equation}\label{epsilon1}
%     \epsilon<\min\{(\eta_0-\delta_0)/3,\nu\}
%   \end{equation}
% and fix it, then let
%   \[L:=e^{(\eta_0-\delta_0-\epsilon)}>1\]
% Pick $K$ big enough in Theorem \ref{omega3} to be such that
%   \[(3M)^{\frac{1}{K}}<L\]
% say, there exists $ \sigma>0$,
% \begin{equation}\label{K}
% (3M)^{\frac{1}{K}}\leq L-\sigma<L
% \end{equation}
Then for the left hand side of \eqref{last}, there are three cases:
\begin{enumerate}
  \item both $|-n-y_{1}|>\frac{n}{K}$ and $|n-y_{2}|>\frac{n}{K}$
  \item one of them is large, say $|-n-y_{1}|>\frac{n}{K}$ while $|n-y_{2}|\leq\frac{n}{K}$
  \item both small.
\end{enumerate}

For $(1)$,
\[
\frac{1}{2n}e^{(\eta_0-\delta_0+\gamma(E_j)-\epsilon)(2n+1)}\leq e^{2n(\gamma(E_j)+\epsilon)}
\]
Since by our choice \eqref{epsilon1},
 $\eta_0-\delta_0+\gamma(E_j)-\epsilon>\gamma(E_j)+\epsilon$, for $n$ large enough, we get a contradiction.

For $(2)$,
\[
  % \begin{aligned}
    \frac{1}{2n}e^{(\eta_0-\delta_0+\gamma(E_j)-\epsilon)(2n+1)}
    \leq e^{(\gamma(E_j)+\epsilon)(2n+1)}(M)^{\frac{n}{K}}
    % \frac{1}{2Cn}e^{(\eta_0-\delta_0-\epsilon)(2n+1)}&\leq e^{\epsilon(2n+1)} L^n\\
    % &\leq e^{\epsilon(2n+1)} e^{(\eta_0-\delta_0-\epsilon)n}\\
    % \frac{1}{2Cn}e^{(\eta_0-\delta_0-\epsilon)(n+1)}
    % &\leq e^{2\epsilon(n+1)}
  % \end{aligned}
\]
is in contradiction with \eqref{epsilon1} and \eqref{K}

For $(3)$, with \eqref{epsilon1} and \eqref{K}
\[
% \begin{aligned}
  \frac{1}{2n}e^{(\eta_0-\delta_0+\gamma(E_j)-\epsilon)(2n+1)}\leq M^{\frac{2n}{K}}\leq (L-\sigma)^{2n}\leq(e^{(\eta_0-\delta_0+\gamma(E_j)-\epsilon)}-\sigma)^{2n},
% \end{aligned}
\]
also a contradiction.

Thus our assumption that $2n+1$ is not  $(\gamma(\tilde{E})-8\epsilon_0,n,\tilde{E},\tilde{\omega})$-regular is false. Theorem \ref{thm2} follows.
\end{proof}
