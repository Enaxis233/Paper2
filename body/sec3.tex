\section{Main Lemmas}

Denote
\begin{equation}\label{B+}
    B_{[a,b],\epsilon}^{+} =\left\{(E,\omega): |P_{[a,b],E,\omega}|\geq e^{(\gamma(E)+\epsilon)(b-a+1)}\right\}
\end{equation}
\begin{equation}\label{B-}
    B_{[a,b],\epsilon}^{-} =\left\{(E,\omega): |P_{[a,b],E,\omega}|\leq e^{(\gamma(E)-\epsilon)(b-a+1)}\right\}\
\end{equation}
and denote $B_{[a,b],\epsilon,E}^{\pm}=\{\omega:(E,\omega)\in B_{[a,b],\epsilon}^{\pm}\}$, $B_{[a,b],\epsilon,\omega}^{\pm}=\{E:(E,\omega)\in B_{[a,b],\epsilon}^{\pm}\}$,
$B_{[a,b],*}=B_{[a,b],*}^+\cup B_{[a,b],*}^-$.

Let $E_{j,(\omega_a,\cdots,\omega_b)}$ be the eigenvalue of $H_{[a,b],\omega}$ with $\omega|_{[a,b]}=(\omega_a,\cdots,\omega_b)$.

% The main technique is to find the proper set $\Omega_0$. Roughly speaking, the idea is: a point $x$ is $(C,k,E,\omega)$-singular means the corresponding pair $(E,\omega)$ is in some "bad" "large deviation set" for operators restricted near $x$, say, $H_{[x-k,x+k]}$. We can then pick proper set $\Omega_1$ such that these bad sets have small measures. Then pick $\Omega_2$ such that $E$ is not only in these bad sets, but also stay very close to eigenvalues of $H_{[x-k,x+k]}$ which are also in these bad sets. In this case, if we pick $\tilde{\omega}\in\Omega_1\cap\Omega_2$ and if $0$ and $2n+1$ are both $(C,n,\tilde{E},\tilde{\omega})$-singular, then $\tilde{E}$ will  be close to both eigenvalues $E_{i,n}$ of $H_{[-n,n]}$ near $0$ and eigenvalues $E_{j,n}$ of $H_{[n+1,3n+1]}$ near $2n+1$. So now $\tilde{E}$ is in bad sets for both intervals and are closed to both e.v., where each e.v. only belongs to bad sets for their own intervals. Which leads to a contradiction because the bad sets are so small that we can pick $\Omega_3$ by Borel Cantelli, such that eventually all the eigenvalues of $[n+1,3n+1]$ can't stay in bad set for $[-n,n]$,

% The main idea is that a point $x$ is $(C,k,E,\omega)$-singular means somehow the pair $(E,\omega)$ is in some "bad" "large deviation set" for operators restricted near $x$, say, $H_{[x-k,x+k]}$.(see lemma1) However we can get from large deviation theorem that these sets are small, so both $E$ and eigenvalues of $H_{[x-k,x+k]}$, which also lies in these bad sets, will eventually get closed to each other due to some nice properties of these sets.


% \begin{remark}
%   If we denote
% \[
% \begin{split}
%     B_{[a,b],\epsilon}=&\left\{(E,\omega): |P_{[a,b],E,\omega}|\geq e^{(\gamma(E)+\epsilon)(b-a+1)}\right\} \\
%     &\bigcup \left\{(E,\omega): |P_{[a,b],E,\omega}|\leq e^{(\gamma(E)-\epsilon)(b-a+1)}\right\}
% \end{split}
% \]
%   we find that
%   \[
%     B_{[a,b],\epsilon}\subseteq\left\{(E,\omega):\left\vert\frac{1}{b-a+1} \log\Vert P_{[a,b],E,\omega}\Vert-\gamma(E)\right\vert \geq\epsilon \right\}
%   \]
%   This is the "bad" set. And Large deviation theorem gives us the estimates that for all $E,a,b$
% \begin{equation}\label{ldt}
%   P(\{\omega:(E,\omega)\in B_{[a,b],\epsilon}\})\leq e^{-\eta(b-a+1)}
% \end{equation}
% Moreover, we denote
% Denote
% \begin{equation}\label{B+}
%     B_{[a,b],\epsilon}^{+} =\left\{(E,\omega): |P_{[a,b],E,\omega}|\geq e^{(\gamma(E)+\epsilon)(b-a+1)}\right\}
% \end{equation}
% \begin{equation}\label{B-}
%     B_{[a,b],\epsilon}^{-} =\left\{(E,\omega): |P_{[a,b],E,\omega}|\leq e^{(\gamma(E)-\epsilon)(b-a+1)}\right\}\
% \end{equation}
% and denote $B_{[a,b],\epsilon,E}^{\pm}=\{\omega:(E,\omega)\in B_{[a,b],\epsilon}\}^{\pm}$.
Large deviation theorem gives us the estimate that for all $E, a, b,\epsilon$
\begin{equation}\label{ldt}
P(B_{[a,b],\epsilon,E}^\pm)\leq e^{-\eta(b-a+1)}
\end{equation}

Assume $\epsilon=\epsilon_0<\frac{1}{8}\nu$ is fixed for now, so we omit it from the notations until Lemma \ref{omega3}. $\eta_0=\eta(\epsilon_0)$ is the corresponding parameter from Lemma \ref{ldt lemma}
\begin{lemma}\label{lemma1}
For $n \geq 2$, if $x$ is $(\gamma(E)-8\epsilon_0,n,E,\omega)$-singular, then
\[(E,\omega)\in B_{[x-n,x+n]}^-\cup B_{[x-n,x]}^+\cup B_{[x,x+n]}^+
\]
\end{lemma}
\begin{remark}
  Note that from \eqref{ldt}, for all $E,x,n\geq 2$,
  \[
    P(B_{[x-n,x+n],E}^-\cup B_{[x-n,x],E}^+\cup B_{[x,x+n],E}^+)\leq 3e^{-\eta_0 (n+1)}
  \]
\end{remark}
\begin{proof}
Follows imediately from the definition of singularity and \eqref{A}.
% Assume not, then
% \[
%   \left\{
%   \begin{aligned}
%      & |P_{[x-n,x+n],E,\omega}|\geq e^{(\gamma(E)+\epsilon_0)(2n+1)} \\
%      & |P_{[x-n,x],E,\omega}|\leq e^{(\gamma(E)-\epsilon_0)(n+1)} \\
%      & |P_{[x,x+n],E,\omega}|\leq e^{(\gamma(E)-\epsilon_0)(n+1)}
%   \end{aligned}
%   \right.
% \]
% So we can estimate
% \[
%   \begin{aligned}
%     \left\vert G_{[x-n,x+n],E,\omega}(x,x-n)\right\vert
%     &=  \frac{\left\vert P_{[x,x+n],E,\omega}\right\vert}{\left\vert P_{[x-n,x+n],E,\omega}\right\vert}\\
%     &\leq  \frac{e^{(\gamma(E)+\epsilon_0)(n+1)}}{e^{(\gamma(E)-\epsilon_0)(2n+1)}}\\
%     &\leq  e^{-\gamma(E)(n)+\epsilon_0(3n+2)}\\
%     &\leq  e^{-(\gamma(E)-8\epsilon_0)n}
%   \end{aligned}
% \]
% Similar for $G_{[x-n,x+n],E,\omega}(x,x+n)$.
% Thus $x$ is $(\gamma(E))-8\epsilon_0, n, E,\omega)$-regular, contradiction.
\end{proof}

% By Theorem \ref{thm2},
% From lemma \ref{lemma1}, we get our first restriction on $\Omega$ to make the bad sets have small measure.
Now we will use the following three lemmas to find the proper $\Omega_0$ for Theorem \ref{thm2}.
\begin{lemma}\label{omega1}
  Let $0<\delta_0<\eta_0$. For a.e. $\omega$ (we denote this set as $\Omega_1$), there exists $ N_1=N_1(\omega)$, such that for every $ n>N_1$,
  \[
  \max\{m(B_{[n+1,3n+1],\omega}^-),m(B_{[-n,n],\omega}^-)\}\leq e^{-(\eta_0-\delta_0)(2n+1)}
  \]
\end{lemma}

\begin{proof}
  By \eqref{ldt},
  % for every $ E\in I=\sigma$,which is compact, $P(B_{[n+1,3n+1],E}^-)\leq e^{-\eta_0(2n+1)}$ and $P(B_{[-n,n],E}^-)\leq e^{-\eta_0(2n+1)}$
\[
m\times\mathbb{P}(B_{[n+1,3n+1]}^-)\leq m(\sigma)e^{-\eta_0(2n+1)}
\]
\[
m\times\mathbb{P}(B_{[-n,n]}^-)\leq m(\sigma)e^{-\eta_0(2n+1)}
\]
  If we denote
    \[
      \Omega_{\delta_0,n,+}=\left\{\omega:m(B_{[n+1,3n+1],\omega}^-)\leq e^{-(\eta_0-\delta_0)(2n+1)}\right\}
    \]
    \[
    \Omega_{\delta_0,n,-}=\left\{\omega:m(B_{[-n,n],\omega}^-)\leq e^{-(\eta_0-\delta_0)(2n+1)}\right\},
    \]

  We have Tchebyshev,
  \[
      \mathbb{P}(\Omega_{\delta_0,n,\pm}^c)
      % &\leq e^{(\eta_0-\delta_0)n}\int_{\Omega} m(B_{[n+1,3n+1],\omega})dP\omega\\
      % &=    e^{(\eta_0-\delta_0)n}\int_I P(B_{[n+1,3n+1],E})dx\\
      % &\leq e^{(\eta_0-\delta_0)n}m(I)e^{-\eta_0(2n+1)}\\
      \leq m(\sigma)e^{-\delta_0(2n+1)}.
  \]
By Borel-Cantelli lemma, we get for $a.e.~\omega$,
\[
\max\{m(B_{[n+1,3n+1],\omega}^-),m(B_{[-n,n],\omega}^-)\}\leq e^{-(\eta_0-\delta_0)(2n+1)},
\]
for $n>N_1(\omega)$.
\end{proof}
\begin{remark}\label{N1}
  Note that we can actually shift the operator and use center point $l$ instead of $0$. Then we will get $\Omega_1(l)$ instead of $\Omega_1$, $N_1(l,\omega)$ instead of $N_1(\omega)$. And if we pick $N_1(l,\omega)$ in the theorem as the smallest interger satisfying the conclusion, we can estimate when will $N_1(l,\omega)\leq \ln^2 |l|$, which is very useful in the proof for dynamical localization in section 6.
  In fact, $\mathbb{P}\{\omega: N_1(l,\omega)>\ln^2 |l|\}\leq C' e^{-\delta_0(2|\ln^2|l|+1)}$, By Borel-Cantelli, for $a.e.\omega$,(We denote this set as $\Omega_{N_1}$,) there exists $L_1(\omega)$, such that for any $|l|>L_1(\omega)$, $N_1(l,\omega)\leq \ln^2|l|$.
\end{remark}
The next results follows from \cite{craig1983subharmonicity}:
\begin{thm}[Craig-Simon]\label{CS}
  For a.e.$\omega$(denote as $\Omega_2$), for all $E$, we have
  \begin{equation}\label{C}
    % \begin{aligned}
      \max\left\{\varlimsup_{n\to\infty} \frac{ \log\Vert T_{[-n,0],E,\omega}\Vert}{n+1}, \varlimsup_{n\to\infty} \frac{\log\Vert T_{[0,n],E,\omega}\Vert}{n+1}\right\}\leq\gamma(E)
  \end{equation}
  \begin{equation}\label{D}
    \max\left\{\varlimsup_{n\to\infty} \frac{\log\Vert T_{[n+1,2n+1],E,\omega}\Vert}{n+1}, \varlimsup_{n\to\infty} \frac{\log\Vert T_{[2n+1,3n+1],E,\omega}\Vert}{n+1}\right\}\leq\gamma(E)
  \end{equation}
\end{thm}
\begin{remark}
  \eqref{C} is a direct reformulation of Craig-Simon, while \eqref{D} follows by exactly the same proof.
\end{remark}
\begin{cor}\label{omega2}
  for every $ \omega\in\Omega_2$, for every $E$, there exists $N_2=N_2(\omega,E)$, such that for every $ n>N_2$,
\[
\begin{aligned}
&\max\{\Vert T_{[-n,0],E,\omega}\Vert, \Vert T_{[0,n],E,\omega}\Vert\}<e^{(\gamma(E)+\epsilon)(n+1)}\\ &\max\{T_{[n+1,2n+1],E,\omega}\Vert, \Vert T_{[2n+1,3n+1],E,\omega}\Vert\}<e^{(\gamma(E)+\epsilon)(n+1)}
\end{aligned}
\]

  % \[
  %   \begin{aligned}
  %     &\Vert T_{[-n,0],E,\omega}\Vert< e^{(\gamma(E)+\epsilon)(n+1)}\\
  %     &\Vert T_{[0,n],E,\omega}\Vert< e^{(\gamma(E)+\epsilon)(n+1)}\\
  %     &\Vert T_{[n+1,2n+1],E,\omega}\Vert< e^{(\gamma(E)+\epsilon)(n+1)}\\
  %     &\Vert T_{[2n+1,3n+1],E,\omega}\Vert< e^{(\gamma(E)+\epsilon)(n+1)}
  %   \end{aligned}
  % \]
\end{cor}
% \begin{remark}
%   Basically speaking, the only difference from Theorem 1.5 in \cite{craig1983subharmonicity} is that we are considering restrictions on some different box-sequences, for example $\{[n+1,2n+1]\}$, instead of the original boxes $\{[0,n]\}$. However, by\ref{LE}, $\gamma(E)$ keeps constant under $\{[n+1,2n+1]\}$, so subharmonic. While $\bar{\gamma_(E)}$ as limsup of $\gamma_(E)_[n+1,2n+1]$ is still submean since $\gamma_(E)_[n+1,2n+1]$ are submean. By properties of submean and subharmonic, together with Fustenberg Theorem and Fubini, we can get the results.
% \end{remark}
% \begin{proof}
%   Only prove $[n+1,2n+1]$ case.\\
%   Claim:
%   \begin{enumerate}
%     \item $\gamma(E)$ is subharmonic.
%     \item $\varlimsup_{n\to\infty} \frac{1}{n+1} \log\Vert T_{[n+1,2n+1],E,\omega}\Vert$ is submean
%   \end{enumerate}
% \end{proof}
\begin{lemma}\label{omega3}
   Let $\epsilon>0,K>1$, For a.e.$\omega$(We denote this set as $\Omega_3=\Omega_3(\epsilon,K)$), there exists $ N_3=N_3(\omega)$, so that for every $ n>N_3$, for every $ E_{j,(\omega_{n+1},\cdots,\omega_{3n+1})}$, for every $ y_1,y_2$ satisfying $-n\leq y_1\leq y_2\leq n$,  $\abs{-n-y_1}\geq\frac{n}{K}$, and $\abs{n-y_2}\geq\frac{n}{K}$,
 we have $E_{j,(\omega_{n+1},\cdots,\omega_{3n+1})}\notin B_{[-n,y_1],\epsilon,\omega}\cup B_{[y_2,n],\epsilon,\omega}$.
\end{lemma}
\begin{remark}
  Note that $\epsilon$ and $K>0$ are not fixed yet, we're going to determine them later in section 4.
\end{remark}
\begin{proof}
% In order to use Borel-Cantelli, one need to estimate
Let $\bar{P}$
% \[
%   \bar{P}=P\left(\bigcup\limits_{y_1,y_2}\bigcup\limits_{j=1}^{2n+1}
%   B_{[-n,y_1],\epsilon,E_{j,(\omega_a,\cdots,\omega_b)}\cup B_{[y_2,n],\epsilon,E_{j,(\omega_{n+1},\cdots,\omega_{3n+1})}\right)
% \]
be the probability that there are some $y_1, y_2, j$ with \[
E_{j,(\omega_{n+1},\cdots,\omega_{3n+1})}\in B_{[-n,y_1],\epsilon,\omega}\cup B_{[y_2,n],\epsilon,\omega}.
\] Note that for any fixed $\omega_c,\cdots,\omega_d $, with $[c,d]\cap[a,b]=\emptyset$, by independence,
\[
\mathbb{P}(B_{[a,b],\epsilon,E_{j,(\omega_c,\cdots,\omega_d)}})=\mathbb{P}_{[a,b]}(B_{[a,b],\epsilon,E_{j,(\omega_c,\cdots,\omega_d)}})\leq e^{-\eta_0(b-a+1)}
\]
Applying to $[a,b]=[-n,y_1]$ or $[y_2,n]$, $[c,d]=[n+1,3n+1]$ and integrating over $\omega_{-n},\cdots,\omega_{y_1}$ or $\omega_{y_2},\cdots,\omega_{n}$, we get
\[
\mathbb{P}(B_{[-n,y_1],\epsilon,E_{j,(\omega_{n+1},\cdots,\omega_{3n+1})}}\cup B_{[y_2,n],\epsilon,E_{j,(\omega_{n+1},\cdots,\omega_{3n+1})}}) \leq 2e^{-\eta_0(\frac{n}{K}+1)},
\]
so
\[
\bar{\mathbb{P}}\leq(2n+1)^3 2e^{-\eta_0(\frac{n}{K}+1)}
\]
Thus by Borel-Cantelli, we can get the result.

% where $y_1,y_2$ satisfy assumptions above. Denote it by $\bar{P}$. Consider
% \[
%   \begin{aligned}
%     P\left(B_{[y_2,n],\epsilon,E_{j,(\omega_{n+1},\cdots,\omega_{3n+1})}\right)
%     &= \int_\Omega \chi_{B_{[y_2,n],\epsilon,E_{j,(\omega_{n+1},\cdots,\omega_{3n+1})}} dP\omega\\
%     &= \int_{S^{2n+1}}\left(\int_{\tilde{\Omega}} \chi_{B}~ d\tilde{\mu}\right)d\mu^{2n+1}(\omega_{n+1},\cdots,\omega_{3n+1})\\
%     &=\int_{S^{2n+1}} \tilde{P}(\tilde{B}_{[y_2,n],\epsilon,E_{j,(\omega_{n+1},\cdots,\omega_{3n+1})})d\mu^{2n+1}(\omega_{n+1},\cdots,\omega_{3n+1})
%   \end{aligned}
% \]
% where for $\tilde{\Omega}$ and $\tilde{\mu}$, one take away the $[n+1,3n+1]$ terms from $\Omega$ and $\mu^{\mathbb{Z}}$
% However, for any fixed $E$, $B_{[y_2,n],\epsilon,E}$ is of the form
% \[
%   \left(\bigotimes\limits_{i\in[y_2,n]}S\right) \times {B'_{[y_2,n],\epsilon}}
% \]
% where
% \[
%   B'_{[y_2,n],\epsilon,E}=\left\{\omega|_{[y_2,n]}:\omega\in B_{[y_2,n],\epsilon,E}\right\}
% \]
% So,
% \[
%   \begin{aligned}
%     P(B_{[y_2,n],\epsilon,E})
%     &= \int_{S^{2n+1}}\left(\int_{\tilde{\Omega}} \chi_{B_{[y_2,n],\epsilon,E}}~ d\tilde{\mu}\right)d\mu^{2n+1}(\omega_{n+1},\cdots,\omega_{3n+1})\\
%     &=\tilde{P}(\tilde{B}_{[y_2,n],E})\times 1\times\cdots\times 1\\
%     &=\tilde{P}(\tilde{B}_{[y_2,n],E})
%   \end{aligned}
% \]
% Br \eqref{ldt},
% \[
% \tilde{P}(\tilde{B}_{[y_2,n],\epsilon,E})\leq Ce^{-\eta\abs{n-y_2}},\quad \forall E
% \]
% So
% \[
% P\left(B_{[y_2,n],\epsilon,E_{j,(\omega_{n+1},\cdots,\omega_{3n+1})}\right)\leq Ce^{-\abs{n-y_2-}}\leq Ce^{-n/K}
% \]
% \[
% \bar{P}\leq C(2n+1)^3e^{-n/K}
% \]
% The sum over n is finite, use Borel-Cantelli, we can get the result.
% By \eqref{ldt}, for any $E$, $P(B_{[a,b],E})<e^{-\eta_0-\delta_0(b-a+1)}$. Since $B_{[a,b],E}$ is a cylinder set that depends only on $\omega_i, i\in[a,b]$, we have that for any $[c,d]\cap[a,b]=\emptyset$,
% \[
%   P(B_[a,b],E|\{\omega: E=E_{j,[c,d],\omega}\})=P(B_[a,b],E)
% \]
% Integrating over  ?
\end{proof}
\begin{remark}\label{N3}
 Similar to remark \ref{N1}, we can get $\Omega_3(l)$, $N_3(l,\omega)$ instead, and get that for $a.e.\omega$, (We denote this set as $\Omega_{N_3}$,)  there exists $L_3(\omega)$, such that for any $|l|>L_3$, $N_3(l,\omega)\leq \ln^2 |l|$.
\end{remark}
